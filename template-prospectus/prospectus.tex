\documentclass{proc}

\begin{document}

\title{Prospectus Template}

\author{You, Me}

\maketitle

\section{Introduction}

% Parallel Coordinates is a great choice when it comes to visualizing multi-variate numerical data. It can be used effectively in many domains to visualize how certain attributes align with other attributes. It has been used  to visualize rankings of candidates 

Parallel Coordinates is a great choice when it comes to visualizing multi-variate numerical data. It can also be used to visualize various ranking problems like change in rankings of football clubs over time, or the ranking of apps in playstore over time. It can also be used in ranking scenarios like group decision making where decision makers build a consensus ranking that best represents the collection of base rankings.

With parallel coordinates in the ranking problem, it is easy to see how the individual candidate's position is ranked by different rankers including consensus ranking. However, fails to explain by how much the consensus ranking agrees or disagrees with the individual rankings. This is because of the main limitation of the parallel coordinates: Comparison is limited to adjcent axes. This problem has been tackled by Parallel Coordinates Plot with Color-coded Stacked Histograms \cite{php}. However, if there are large number of base rankings (axes), we would still need to scroll through all base rankings to see how consensus aligns with individual base rankings. In terms of parallel coordinates, it is not possible to compare an axis with all other axes combined.

To solve this, we attempt to integrate parallel coordinates plot with other idioms like box plot or strip plot. We hope that our design will make it possible to analyze the agreement of base rankings with consensus ranking. 

% 
% Cite hindalong.

\section{One-sentence description}
Combining parallel coordinates with other idioms like strip plot or box plot, we attempt to make analysis of consensus ranking easier.


\section{Project Type}
Development

\section{Audience} 
\begin{quote}
\textit{Who is the audience for this project? 
How does it meet their needs? 
What happens if their needs remain unmet?}
\end{quote}

- Any domain where consensus ranking needs to be generated

- eg: scholarships, ranking of colleges and universities, admissions by admissions committee, competitions 

\section{Approach}
\subsection{Details}
\begin{quote}
\textit{What is your approach?}
\end{quote}



\subsection{Evidence for Success}
\begin{quote}
\textit{Why do you think it will work?} 
\end{quote}

issues and how the lines or box plot will explain degree of consensus

\section{Best-case Impact Statement}
\begin{quote}
\textit{In the best-case scenario, what would be the impact statement (conclusion statement) for this project? \cite{wijk2005value, pike2009science}}
\end{quote}

explain degree of consensus, understandability should be maximum.

\section{Major Milestones}

1. Generate a dataset with large number of rankers and candidates; and multiple consensus rankings;
2. Implementation of visualization... (hover to view each base ranking)
3. Allow user to manipulate consensus ranking and see how the degree of agreement or disagreement changes
4. Fish eye view for large dataset exploration

\section{Obstacles}

\subsection{Major obstacles} % (if these fail, the project is over)
2. interactive visualization of agreement on each candidate

\subsection{Minor obstacles}
1. Fisheye view
2. Asthetic aspect
3. manipulate ranking.

\section{Resources Needed}
\begin{quote}
\textit{What additional resources do you need to complete this project?}
\end{quote}

1. Code to generate large number of base rankings which have atleast some agreement 
2. 

\section{5 Related Publications}
\begin{quote}
\textit{List 5 major publications that are most relevant to this project, and how they are related (sample citation \cite{wijk2005value}).}
\end{quote}

\section{Define Success}
\begin{quote}
\textit{What is the minimum amount of work necessary for this work be publishable?}
\end{quote}

\bibliographystyle{abbrv}
\bibliography{prospectus}
\end{document}
